\documentclass[12pt,oneside,a4paper]{article}
\usepackage{amsmath}
\usepackage{amssymb}
\usepackage[brazilian]{babel}
\usepackage[utf8]{inputenc}
\usepackage[T1]{fontenc}
\usepackage[margin=2cm]{geometry}
% \usepackage[parfill]{parskip} % no indentation
\setlength{\parskip}{12pt} % redimensiona o tamanho do espaçamento entre parágrafos

\begin{document}
5. Calcule o resultado da seguinte soma: 
\begin{equation*}
\sum_{n=1}^{\infty}\frac{2}{4n^{2}+8n+3}
\end{equation*}

Solução:

Fatore o denominador:
\begin{equation*}
4n^{2}+8n+3=(2n+1)(2n+3)
\end{equation*}

Separe em frações parciais:
\begin{equation*}
\frac{2}{4n^{2}+8n+3}=\frac{A}{2n+1}+\frac{B}{2n+3}
\end{equation*}

Ao resolver:
\begin{equation*}
2=A(2n+3)+B(2n+1)
\end{equation*}
\begin{equation*}
2=(2A+2B)n+(3A+B)
\end{equation*}

É possível formar o sistema linear:
\begin{equation*}
\begin{cases}
2A + 2B = 0 \\
3A + B = 2
\end{cases}
\end{equation*}

De onde se encontra que: $A=1$ e $B=-1$

Assim,
\begin{equation*}
\frac{2}{4n^{2}+8n+3}=\frac{1}{2n+1}-\frac{1}{2n+3}
\end{equation*}

A soma torna-se:
\begin{equation*}
\sum_{n=1}^{\infty}\left(\frac{1}{2n+1}-\frac{1}{2n+3}\right)
\end{equation*}

Calcule os primeiros termos:
\begin{equation*}
=\left(\frac{1}{3}-\frac{1}{5}\right)+\left(\frac{1}{5}-\frac{1}{7}\right)+\left(\frac{1}{7}-\frac{1}{9}\right)+\cdots
\end{equation*}

Todos os termos se anulam, exceto pelo primeiro. Portanto:
\begin{equation*}
=\frac{1}{3}
\end{equation*}
\pagebreak

18. Calcule
\begin{equation*}
\int_{1}^{2}\frac{e^{x}\left(x-1\right)}{x\left(x+e^{x}\right)}\;dx
\end{equation*}

Solução:

Note que
\begin{equation*}
\frac{d}{dx}\ln{x}=\frac{1}{x}
\end{equation*}

E do termo em parênteses do denominador,
\begin{equation*}
\frac{d}{dx}\ln{\left(x+e^{x}\right)}=\frac{1+e^{x}}{x+e^{x}}
\end{equation*}

A diferença entre esses diferenciais resulta em:
\begin{equation*}
\frac{d}{dx}\left(\ln{\left(x+e^{x}\right)}-\ln{x}\right)=\frac{1+e^{x}}{x+e^{x}}-\frac{1}{x}
\end{equation*}

Desenvolva em denominador comum:
\begin{equation*}
=\frac{x\left(1+e^{x}\right)-\left(x+e^{x}\right)}{x\left(x+e^{x}\right)}
\end{equation*}

Simplifique o numerador:
\begin{equation*}
x+xe^{x}-x-e^{x}=xe^{x}-e^{x}=e^{x}\left(x-1\right)
\end{equation*}

Portanto,
\begin{equation*}
\frac{d}{dx}\left(\ln{\left(x+e^{x}\right)}-\ln{x}\right)=\frac{e^{x}\left(x-1\right)}{x\left(x+e^{x}\right)}
\end{equation*}

Pela propriedade da diferença entre logaritmos,
\begin{equation*}
\frac{d}{dx}\left(\ln{\frac{x+e^{x}}{x}}\right)=\frac{e^{x}\left(x-1\right)}{x\left(x+e^{x}\right)}
\end{equation*}

Ao integrar,
\begin{equation*}
\int_{1}^{2}\frac{e^{x}\left(x-1\right)}{x\left(x+e^{x}\right)}\;dx=\left[\ln{\frac{x+e^{x}}{x}}\right]_{1}^{2}
\end{equation*}

\begin{equation*}
=\ln{\frac{2+e^{2}}{2}}-\ln{\left(1+e\right)}=\ln{\frac{2+e^{2}}{2\left(1+e\right)}}=\ln{\frac{2+e^{2}}{2+2e}}
\end{equation*}

\end{document}